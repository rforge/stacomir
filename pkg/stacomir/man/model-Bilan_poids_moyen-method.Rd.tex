% Generated by roxygen2: do not edit by hand
% Please edit documentation in R/Bilan_poids_moyen.r
\docType{methods}
\name{model,Bilan_poids_moyen-method}
\alias{model,Bilan_poids_moyen-method}
\alias{model.Bilan_poids_moyen}
\alias{model.bilPM}
\title{model method for Bilan_poids_moyen' 
this method uses samples collected over the season to model the variation in weight of
glass eel or yellow eels.}
\usage{
model(object,model.type=c("seasonal","seasonal1","seasonal2","manual"),silent=FALSE)
}
\arguments{
\item{object}{An object of class \link{Bilan_pois_moyen-class}}

\item{model.type}{default "seasonal", "seasonal1","seasonal2","manual".}
}
\description{
model method for Bilan_poids_moyen' 
this method uses samples collected over the season to model the variation in weight of
glass eel or yellow eels.
}
\details{
Depending on model.type several models are produced
\itemize{
		\item{model.type="seasonal".}{ The simplest model uses a seasonal variation, it is
				fitted with a sine wave curve allowing a cyclic variation 
				w ~ a*cos(2*pi*(doy-T)/365)+b with a period T. The julian time d0 used is this model is set
				at zero 1st of November d = d + d0; d0 = 305.}
	\item{model.type="seasonal1".}{ A time component is introduced in the model, which allows
			for a long term variation along with the seasonal variation. This long term variation is
			is fitted with a gam, the time variable is set at zero at the beginning of the first day of observed values.
			The seasonal variation is modeled on the same modified julian time as model.type="seasonal"
			but here we use a cyclic cubic spline cc, which allows to return at the value of d0=0 at d=365.
			This model was considered as the best to model size variations by Diaz & Briand in prep. but using a large set of values
			over years.}
\item{model.type="seasonal2"}{The seasonal trend in the previous model is now modelled with a sine
			curve similar to the sine curve used in seasonal.  The formula for this is \eqn{sin(\omega vt) + cos(\omega vt)}{sin(omega vt) + cos(omega vt)}, 
		where vt is the time index variable \eqn{\omega}{omega} is a constant that describes how the index variable relates to the full period
			(here, \eqn{2\pi/365=0.0172}{2pi/365=0.0172}. The model is written as following \eqn{w~cos(0.0172*doy)+sin(0.0172*doy)+s(time).}}
	\item{model.type="manual"}{ The dataset don (the raw data), coe (the coefficients already present in the
			database, and newcoe the dataset to make the predictions from, are written to the environment envir_stacomi. 
			please see example for further description on how to fit your own model, build the table of coefficients,
			and write it to the database.}	
}
}
\author{
Cedric Briand \email{cedric.briand"at"eptb-vilaine.fr}
}

